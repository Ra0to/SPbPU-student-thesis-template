\chapter{Код пакета}\label{appendix-1}							% Заголовок
%\addcontentsline{toc}{chapter}{Second call for chapters to participate in the book Machine learning in analysis of biomedical and socio-economic data}	% Добавляем его в оглавление
\section{Код установочного скрипта}\label{app-1}
Файл install\_script.sql:

%\begin{lstlisting}
\begin{verbatim}
--CREATE ERROR_TYPES TABLE
create table errm$error_types (
code number(5) primary key,
name varchar2(100 char) unique not null,
info varchar2(2000 char)
);

create sequence errm$errors_type_seq start with 4 increment by 1;

insert into errm$error_types values (0, 'error', 'Exception');
insert into errm$error_types values (1, 'warning', 'Warning');
insert into errm$error_types values (2, 'log', 'Log information');
insert into errm$error_types values (3, 'critical_error', 'Exceptions that require additional actions');

--CREATE ERRORS TABLE
create table errm$errors (
code number(10) primary key,
name varchar2(100) unique not null,
type number(5) default 0 references errm$error_types not null,
info varchar2(2000) not null,
usage_comment varchar2(1000),
create_date date,
create_by varchar2(100)
);
/

create sequence errm$errors_code_seq start with 5 increment by 1;

create or replace trigger errm$additional_info_biu_trg
before insert or update
on errm$errors
for each row
begin
:new.name := lower(:new.name);
if :old.create_date is null then
:new.create_date := sysdate;
:new.create_by := SYS_CONTEXT ('USERENV', 'SESSION_USER');
end if;
end errm$additional_info_biu_trg;

insert into errm$errors values (0, 'default_error', 0, 'Error has occurred', 'Simple error for tests. Do not use it in production!', null, null);
insert into errm$errors values (1, 'not_implemented', 0, 'Code not implemented', 'Use when code not yet implemented.', null, null);
insert into errm$errors values (2, 'default_warning', 1, 'Warning information', 'Simple warning for tests. Do not use it in production!', null, null);
insert into errm$errors values (3, 'default_log', 2, 'Log information', 'Simple log message for tests. Do not use it in production!', null, null);
insert into errm$errors values (4, 'critical_error', 3, 'Critical error has occurred', 'Simple critical error for tests. Do not use it in production!', null, null);

--CREATE LOG TABLE
create table errm$log (
id number(38) primary key,
code number(10) references errm$errors not null,
timestamp timestamp not null,
message varchar2(1000),
context varchar2(2000)
);

create sequence errm$log_seq start with 1 increment by 1;

create table errm$parameters (
name varchar2(50) primary key,
value varchar2(1000)
);

insert into errm$parameters values ('MAIL', 'mikl181@yandex.ru');

create or replace trigger errm$parameters_biu_trg
before insert or update
on errm$parameters
for each row
begin
:NEW.name := upper(:NEW.name);
:NEW.value := lower(:NEW.value);
end errm$parameters_biu_trg;

create table errm$feature_flags (
name varchar2(50) primary key,
value char(1) default '0'
);

insert into errm$feature_flags values ('ALWAYS_ADD_DEFAULT_CONTEXT', '1');
insert into errm$feature_flags values ('MAIL_CRITICAL_ERRORS', '0');
insert into errm$feature_flags values ('LOG_ERRORS_TO_FILE', '1');
insert into errm$feature_flags values ('LOG_ERRORS_TO_TABLE', '1');

create or replace trigger errm$feature_flags_biu_trg
before insert or update
on errm$feature_flags
for each row
begin
:NEW.name := upper(:NEW.name);

if (:new.value not in ('0', '1')) then
raise_application_error(-20000, 'Invalid data');
end if;
end errm$feature_flags_biu_trg;

create or replace view errm$most_often_errors as
with
total_count as (
select count(*) cnt
from ERRM$LOG
),
error_count as (
select code, count(*) cnt
from ERRM$LOG
group by code
)
select ec.CODE, ec.cnt, e.name, e.TYPE, et.NAME type_name, e.INFO, e.USAGE_COMMENT
from error_count ec left join ERRM$ERRORS e on ec.CODE = e.CODE left join ERRM$ERROR_TYPES et on e.TYPE = et.CODE
where cnt > 0.1 * (select cnt from total_count);

create or replace view errm$most_popular_errors as
with
error_count as (
select code, count(*) cnt
from ERRM$LOG
group by code
)
select ec.CODE, ec.cnt, e.name, e.TYPE, et.NAME type_name, e.INFO, e.USAGE_COMMENT
from error_count ec left join ERRM$ERRORS e on ec.CODE = e.CODE left join ERRM$ERROR_TYPES et on e.TYPE = et.CODE
order by ec.cnt desc;

create directory errm$log_dir as 'E:/ERRM';

@@errm.pks
@@errm.pkb
\end{verbatim}


\section{Код спецификации пакета}\label{app-2}
Файл errm.pks:

\begin{verbatim}
create or replace package errm
is
--CONSTANTS
feature_mail_critical_errors constant varchar2(50 char) := 'MAIL_CRITICAL_ERRORS';
feature_log_errors_to_file constant varchar2(50 char) := 'LOG_ERRORS_TO_FILE';
feature_log_errors_to_table constant varchar2(50 char) := 'LOG_ERRORS_TO_TABLE';
feature_add_default_context constant varchar2(50 char) := 'ALWAYS_ADD_DEFAULT_CONTEXT';

parameter_mail constant varchar2(50 char) := 'MAIL';

--TYPES
subtype t_err_code is pls_integer;
subtype t_err_name is varchar2(100 char);
subtype t_err_type is pls_integer;

--PARAMETERS WORK
function get_feature_value(p_feature_name in varchar2) return boolean;
procedure set_feature_value(p_feature_name in varchar2, p_value in BOOLEAN);

function get_parameter_value(p_param_name in varchar2) return varchar2;
procedure set_parameter_value(p_param_name in varchar2, p_value in varchar2);

--HELPERS
function get_error_code(p_name in t_err_name) return t_err_code;

--PROCEDURE AND FUNCTIONS
procedure register_error(p_code in t_err_code := null, p_name in t_err_name, p_type t_err_type := 0, p_info in varchar2, p_comment in varchar2 := null);

procedure add_default_context;
procedure add_context(p_name in varchar2, p_value in varchar2);
procedure clear_context;

procedure raise(p_code in t_err_code, p_log_enabled in boolean := TRUE);
procedure raise(p_name in t_err_name, p_log_enabled in boolean := TRUE);

--SILENT RAISES
procedure sraise(p_code in t_err_code);
procedure sraise(p_name in t_err_name);

--LOG WORK
procedure clear_log_table;
procedure clear_log_file;

--SIMPLE ERRORS RAISE
procedure default_error;
procedure default_log;
procedure default_warning;
procedure critical_error;
procedure not_implemented_exception;

--LOG
procedure log (p_text in varchar2);
procedure nl;
end errm;
/
\end{verbatim}

\section{Код тела пакета}\label{app-3}
Файл errm.pkb:

\begin{verbatim}
create or replace package body errm
is
--CONSTANTS

c_default_error constant pls_integer := -20000;
--New line symbol
c_nl constant char(1) := chr(10);


--TYPES

TYPE t_context IS TABLE OF VARCHAR2(100) INDEX BY VARCHAR2(100);

--VARIABLES

g_context t_context;

--PARAMETERS WORK

function get_feature_value(p_feature_name varchar2)
return boolean
is
l_value errm$feature_flags.value%type;
begin
select value
into l_value
from ERRM$FEATURE_FLAGS
where NAME = upper(p_feature_name);

return l_value = '1';
end get_feature_value;

procedure set_feature_value(p_feature_name varchar2, p_value BOOLEAN)
is
pragma autonomous_transaction;
begin

if p_value then
update ERRM$FEATURE_FLAGS
set value = '1'
where NAME = upper(p_feature_name);
else
update ERRM$FEATURE_FLAGS
set value = '0'
where NAME = upper(p_feature_name);
end if;

if sql%rowcount = 0 then
raise no_data_found;
end if;
commit;
end set_feature_value;

function get_parameter_value(p_param_name varchar2)
return varchar2
is
l_value ERRM$PARAMETERS.value%type;
begin
select value
into l_value
from ERRM$PARAMETERS
where NAME = upper(p_param_name);

return l_value;
end get_parameter_value;

procedure set_parameter_value(p_param_name varchar2, p_value varchar2)
is
pragma autonomous_transaction;
begin
update ERRM$PARAMETERS
set value = lower(p_value)
where NAME = upper(p_param_name);

if sql%rowcount = 0 then
raise no_data_found;
end if;
commit;
end set_parameter_value;


--HELPERS

procedure reset_seq(p_seq_name in varchar2)
is
pragma autonomous_transaction;
l_val number;
begin
--Hack from AskTom for reset sequence
execute immediate 'select ' || p_seq_name || '.nextval from dual' INTO l_val;
execute immediate 'alter sequence ' || p_seq_name || ' increment by -' || l_val || ' minvalue 0';
execute immediate 'select ' || p_seq_name || '.nextval from dual' INTO l_val;
execute immediate 'alter sequence ' || p_seq_name || ' increment by 1 minvalue 0';
end reset_seq;

procedure add_default_context
is
begin
add_context('instance', SYS_CONTEXT ('USERENV', 'INSTANCE_NAME'));
add_context('db_name', SYS_CONTEXT ('USERENV', 'DB_NAME'));
add_context('user_name', SYS_CONTEXT ('USERENV', 'SESSION_USER'));
add_context('module', SYS_CONTEXT ('USERENV', 'MODULE'));
add_context('current_schema', SYS_CONTEXT ('USERENV', 'CURRENT_SCHEMA'));
add_context('host', SYS_CONTEXT ('USERENV', 'HOST'));
end add_default_context;

procedure add_context(p_name in varchar2, p_value in varchar2)
is
begin
g_context(p_name) := p_value;
end add_context;

procedure clear_context
is
begin
g_context.DELETE();
end clear_context;

function get_error_info(p_code in t_err_code)
return errm$errors%rowtype
is
l_error errm$errors%rowtype;
begin
select *
into l_error
from ERRM$ERRORS
where code = p_code;

return l_error;
end get_error_info;

function get_context_string(p_use_tab boolean := true)
return varchar2
is
l_index   varchar2(100);
l_context varchar2(1000);
begin
if get_feature_value(errm.FEATURE_ADD_DEFAULT_CONTEXT) then
add_default_context();
end if;

if g_context.COUNT = 0 then
return null;
end if;

l_index := g_context.FIRST;

while l_index is not null
loop
l_context := l_context || case when p_use_tab then '    ' else '' end || l_index || ': ' || g_context(l_index) || c_nl;
l_index := g_context.next(l_index);
end loop;

return l_context;
end get_context_string;

function get_error_message(p_code in t_err_code)
return varchar2
is
r_error_instance errm$errors%rowtype;
l_error_typename errm$error_types.name%type;
l_context varchar2(2000);
begin
r_error_instance := get_error_info(p_code);

select name
into l_error_typename
from ERRM$ERROR_TYPES
where code = r_error_instance.TYPE;

l_context := get_context_string();

return  r_error_instance.name || ' (#' || p_code ||')' || c_nl ||
r_error_instance.info || c_nl ||
'Type: ' || l_error_typename || c_nl ||
'Context' ||
case
when l_context is null then
' not provided!'
else
': ' || c_nl || l_context
end
|| c_nl;
end get_error_message;


procedure log_to_file(p_code in t_err_code)
is
l_log UTL_FILE.file_type;
begin
if NOT get_feature_value(FEATURE_LOG_ERRORS_TO_FILE) then
return;
end if;

l_log := UTL_FILE.FOPEN('ERRM$LOG_DIR', 'log.txt','A'); -- A - append
UTL_FILE.PUT(l_log, systimestamp || ': ' || get_error_message(p_code));
UTL_FILE.FCLOSE(l_log);
end log_to_file;

procedure log_to_table(p_code in t_err_code)
is
pragma autonomous_transaction;

l_error errm$errors%rowtype;
l_message errm$log.message%type;
l_context errm$log.context%type;
begin
if NOT get_feature_value(FEATURE_LOG_ERRORS_TO_TABLE) then
return;
end if;

l_error := get_error_info(p_code);
l_message := get_error_message(p_code);
l_context := get_context_string(false);

insert into ERRM$LOG values (errm$log_seq.nextval, p_code, systimestamp, l_message, l_context);
commit;
end log_to_table;

procedure log_to_mail(p_code in t_err_code)
is
l_host varchar2(100);
l_db_name varchar2(100);
l_error_instance errm$errors%rowtype;
l_message varchar2(2000);
begin
if NOT get_feature_value(FEATURE_MAIL_CRITICAL_ERRORS) then
return;
end if;
l_error_instance := get_error_info(p_code);

if l_error_instance.TYPE != 4 then
return;
end if;

l_host := SYS_CONTEXT ('USERENV', 'HOST');
l_db_name := SYS_CONTEXT ('USERENV', 'DB_NAME');
l_message := get_error_message(p_code);

UTL_MAIL.send(
sender     => l_db_name || '@' || l_host || '.com',
recipients => get_parameter_value(errm.PARAMETER_MAIL),
subject    => l_error_instance.NAME || ' has occurred on ' || l_db_name || '/' || l_host,
message    => l_message);
end log_to_mail;

procedure log_error(p_code in t_err_code)
is
begin
log_to_table(p_code);
log_to_file(p_code);
log_to_mail(p_code);
end log_error;

function get_error_code(p_name in t_err_name)
return t_err_code
is
l_err_code t_err_code;
begin
select code
into l_err_code
from errm$errors
where lower(name) = lower(p_name);

return l_err_code;
end get_error_code;


--PROCEDURE AND FUNCTIONS

procedure register_error(p_code in t_err_code := null, p_name in t_err_name, p_type t_err_type := 0, p_info in varchar2, p_comment in varchar2)
is
pragma autonomous_transaction;
begin
insert into ERRM$ERRORS
values (NVL(p_code, errm$errors_code_seq.nextval), lower(p_name), p_type, p_info, p_comment, null, null);
commit;
end register_error;


procedure raise(p_code in t_err_code, p_log_enabled in boolean := TRUE)
is
l_message varchar2(2000);
begin
if p_log_enabled then
log_error(p_code);
end if;

l_message := get_error_message(p_code);

clear_context();
raise_application_error(c_default_error, l_message);
end raise;

procedure raise(p_name in t_err_name, p_log_enabled in boolean := TRUE)
is
l_err_code t_err_code;
begin
l_err_code := get_error_code(p_name);
errm.raise(l_err_code, p_log_enabled);
end raise;

procedure sraise(p_code in t_err_code)
is
begin
errm.raise(p_code, FALSE);
end sraise;

procedure sraise(p_name in t_err_name)
is
begin
errm.raise(p_name, FALSE);
end sraise;


--LOG WORK

procedure clear_log_table
is
pragma autonomous_transaction;
begin
execute immediate 'truncate table errm$log';

--We can't drop and recreate our sequence, because package should be compiled again
reset_seq('ERRM$LOG_SEQ');
end clear_log_table;

procedure clear_log_file
is
l_log UTL_FILE.file_type;
begin
l_log := UTL_FILE.FOPEN('ERRM$LOG_DIR', 'log.txt','W');
UTL_FILE.FCLOSE(l_log);
end clear_log_file;

--EASE ERRORS RAISE

procedure default_error
is
begin
errm.raise(0);
end default_error;

procedure not_implemented_exception
is
begin
errm.raise(1);
end not_implemented_exception;

procedure default_warning
is
begin
errm.raise(2);
end default_warning;

procedure default_log
is
begin
errm.raise(3);
end default_log;

procedure critical_error
is
begin
errm.raise(4);
end critical_error;


--LOG

procedure log (p_text in varchar2)
is
begin
dbms_output.put_line(p_text);
end log;

procedure nl
is
begin
dbms_output.new_line;
end nl;
end errm;
/	
\end{verbatim}

\section{Код скрипта для удаления}\label{app-4}
Файл uninstall.sql:

\begin{verbatim}
drop view ERRM$MOST_OFTEN_ERRORS;
drop view ERRM$MOST_POPULAR_ERRORS;

drop trigger errm$additional_info_biu_trg;
drop trigger errm$parameters_biu_trg;
drop trigger errm$feature_flags_biu_trg;

drop table ERRM$PARAMETERS;
drop table ERRM$FEATURE_FLAGS;

drop table ERRM$LOG;
drop table ERRM$ERRORS;
drop table ERRM$ERROR_TYPES;

drop sequence ERRM$ERRORS_CODE_SEQ;
drop sequence ERRM$ERRORS_TYPE_SEQ;
drop sequence ERRM$LOG_SEQ;

drop package ERRM;

drop directory errm$log_dir;
\end{verbatim}

\section{Файл с инструкциями по установке}\label{app-5}
Файл readme.txt:

\begin{verbatim}
Инструкция по установке пакета.
1. Найти в файле install.sql строку
create directory errm$log_dir as 'E:/ERRM';
2. Заменить путь директории на необходимый для хранения фалйов логирования
3. Настроить пакет UTL_MAIL
4. Выполнить скрипт install.sql
5. Прочитать описание параметров и опций
6. Настроить пакет по своему усмотрению при помощи процедур SET_PARAMETER_VALUE и SET_FEATURE_VALUE

Удаление пакета:
1. Запустить скрипт uninstall.sql
\end{verbatim}

\section{Скрипт с примерами использования пакета}\label{app-6}
Файл examples.sql:

\begin{verbatim}
--Example 1. Standard exception raise
declare
not_implemented_exception EXCEPTION;
pragma exception_init(not_implemented_exception, -20000);
begin
raise not_implemented_exception;
end;
/

--Example 2. Define new error and raise
begin
errm.register_error(p_name => 'new_not_implemented', p_info => 'Code not implemented');
errm.raise('new_not_implemented');
end;
/

--Example 3. Raise predefined error
begin
errm.not_implemented_exception;
end;
/

--Example 4. Using context for save information
begin
errm.add_context('local_variable', 'null');
errm.default_error;
end;
/

--Example 5. Silent raise with code
begin
errm.sraise(1);
end;
/

--Example 6. Silent raise with name
begin
errm.sraise('critical_error');
end;
/

--Example 7. Raise with name
begin
errm.raise('critical_error');
end;
/

--Example 8. Raise with error code
begin
errm.raise(0);
end;
/
\end{verbatim}

\section{Скрипт для тестирования пакета}\label{app-7}
Файл test.sql:

\begin{verbatim}
SET SERVEROUTPUT ON;
declare
l_bool_value boolean;
l_varchar_value varchar2(2000);
l_int_value pls_integer;
begin
--Auto test
--No data found
errm.LOG('Test 1');
begin
errm.RAISE('code_implemented');
exception
when no_data_found then
errm.LOG('Correct!');
when others then
errm.LOG('Error!');
end;

errm.LOG('Test 2');
begin
l_bool_value := errm.GET_FEATURE_VALUE('unknown_feature');
exception
when no_data_found then
errm.LOG('Correct!');
when others then
errm.LOG('Error!');
end;

errm.LOG('Test 3');
begin
errm.SET_FEATURE_VALUE('unknown_feature', false);
exception
when no_data_found then
errm.LOG('Correct!');
when others then
errm.LOG('Error!');
end;

errm.LOG('Test 4');
begin
l_varchar_value := errm.GET_PARAMETER_VALUE('unknown_parameter');
exception
when no_data_found then
errm.LOG('Correct!');
when others then
errm.LOG('Error!');
end;

errm.LOG('Test 5');
begin
errm.SET_PARAMETER_VALUE('unknown_parameter', 'unknown_value');
exception
when no_data_found then
errm.LOG('Correct!');
when others then
errm.LOG('Error!');
end;

errm.LOG('Test 6');
begin
l_int_value := errm.GET_ERROR_CODE('unknown_error');
exception
when no_data_found then
errm.LOG('Correct!');
when others then
errm.LOG('Error!');
end;

errm.LOG('Test 7');
begin
errm.RAISE(-1);
exception
when no_data_found then
errm.LOG('Correct!');
when others then
errm.LOG('Error!');
end;

errm.LOG('Test 8');
begin
errm.RAISE(-1, false);
exception
when no_data_found then
errm.LOG('Correct!');
when others then
errm.LOG('Error!');
end;

errm.LOG('Test 9');
begin
errm.RAISE('unknown_error');
exception
when no_data_found then
errm.LOG('Correct!');
when others then
errm.LOG('Error!');
end;

errm.LOG('Test 10');
begin
errm.RAISE('unknown_error', false);
exception
when no_data_found then
errm.LOG('Correct!');
when others then
errm.LOG('Error!');
end;

errm.LOG('Test 11');
begin
errm.SRAISE(-1);
exception
when no_data_found then
errm.LOG('Correct!');
when others then
errm.LOG('Error!');
end;

errm.LOG('Test 12');
begin
errm.SRAISE('unknown_error');
exception
when no_data_found then
errm.LOG('Correct!');
when others then
errm.LOG('Error!');
end;

--Manual test
--Context added to error
--     errm.ADD_DEFAULT_CONTEXT;
--     errm.ADD_CONTEXT('user_context', 'user_value');
--     errm.DEFAULT_ERROR;

--No context
--     errm.SET_FEATURE_VALUE(errm.FEATURE_ADD_DEFAULT_CONTEXT, false);
--     errm.ADD_CONTEXT('user_context', 'user_value');
--     errm.CLEAR_CONTEXT;
--     errm.DEFAULT_ERROR;

--Default context
--     errm.SET_FEATURE_VALUE(errm.FEATURE_ADD_DEFAULT_CONTEXT, true);
--     errm.ADD_CONTEXT('user_context', 'user_value');
--     errm.CLEAR_CONTEXT;
--     errm.DEFAULT_ERROR;

--Table ERRM$LOG should be clean
--     errm.CLEAR_LOG_TABLE;

--Table ERRM$LOG should be clean
--     errm.SET_FEATURE_VALUE(errm.FEATURE_LOG_ERRORS_TO_TABLE, false);
--     errm.DEFAULT_ERROR;

--Correct types for each error
--     errm.DEFAULT_ERROR;
--     errm.DEFAULT_LOG;
--     errm.DEFAULT_WARNING;
--     errm.NOT_IMPLEMENTED_EXCEPTION;
--     errm.CRITICAL_ERROR;

--Error occurred
--     errm.raise('critical_error');
--     errm.RAISE(4);

--Error occurred with no additional log
--     errm.SET_FEATURE_VALUE(errm.FEATURE_LOG_ERRORS_TO_TABLE, true);
--     errm.SRAISE(4);

--Added information to file
--     errm.SET_FEATURE_VALUE(errm.FEATURE_LOG_ERRORS_TO_FILE, true);
--     errm.DEFAULT_LOG;
end;
/
\end{verbatim}
