%%%% Начало оформления заголовка - оставить без изменений !!! %%%%
\input{my_folder/task_settings}	% настройки - начало 
	
				{%\normalfont %2020
						\MakeUppercase{\SPbPU}}\\
				\institute

\par}\intervalS% завершает input

				\noindent
				\begin{minipage}{\linewidth}
				\vspace{\mfloatsep} % интервал 	
				\begin{tabularx}{\linewidth}{Xl}
					&УТВЕРЖДАЮ      \\
					&\HeadTitle     \\			
					&\underline{\hspace*{0.1\textheight}} \Head     \\
					&<<\underline{\hspace*{0.05\textheight}}>> \underline{\hspace*{0.1\textheight}} \thesisYear г.  \\  
				\end{tabularx}
				\vspace{\mfloatsep} % интервал 	
				\end{minipage}

\intervalS{\centering\bfseries%

				ЗАДАНИЕ\\
				на выполнение %с 2020 года 
				%по выполнению % до 2020 года
				выпускной квалификационной работы


\intervalS\normalfont%

				студенту \uline{\AuthorFullDat{} гр.~\group}


\par}\intervalS%
%%%%
%%%% Конец оформления заголовка  %%%%
 	
	
	
\begin{enumerate}[1.]
	\item Тема работы: {\expandafter \ulined \thesisTitle.}
	%\item Тема работы (на английском языке): \uline{\thesisTitleEn.} % вероятно после 2021 года
	\item Срок сдачи студентом законченной работы: \uline{\thesisDeadline.} 
	\item Исходные данные по работе: \uline{
		документация Oracle Database \cite{docOracleDB}, схема HR для тестирования работы из скрипта \cite{HrSchemaScript}, основной литературой является руководство \cite{ferstein}, а также статья \cite{handling-errors}.}%
	\printbibliographyTask % печать списка источников % КОММЕНТИРУЕМ ЕСЛИ НЕ ИСПОЛЬЗУЕТСЯ
	% В СЛУЧАЕ, ЕСЛИ НЕ ИСПОЛЬЗУЕТСЯ МОЖНО ТАКЖЕ ЗАЙТИ В setup.tex и закомментировать \vspace{-0.28\curtextsize}
	\item Содержание работы (перечень подлежащих разработке вопросов):
	\begin{enumerate}[label=\theenumi\arabic*.]
		\item Выявление существующих проблем при обработке исключительных ситуация стандартными способами.
		\item Анализ возможных способов устранения выявленных недостатков.
		\item Анализ существующих пакетов для обработки ошибок в Oracle.
		\item Исследование и разработка методик автоматизации и стандартизации работы с исключительными ситуациями.
		\item Разработка программы на PL/SQL для автоматизации и стандартизации обработки ошибок.
		\item Аппробация разработанного пакета.
	\end{enumerate}
	\item Перечень графического материала (с указанием обязательных чертежей): 
	\begin{enumerate}[label=\theenumi\arabic*.]
		\item Диаграмма архитектуры разрабатываемого пакета.
		\item ER модель используемой для тестирования схемы HR.
	\end{enumerate}	
		\item Консультанты по работе:
		\begin{enumerate}[label=\theenumi\arabic*.] 
		%\item  \uline{\emakefirstuc{\ConsultantExtraDegree}, \ConsultantExtra.} % закомментировать при необходимости, идёт первый по порядку.
		\item \uline{\emakefirstuc{\ConsultantNormDegree}, \ConsultantNorm{} (нормоконтроль).} %	Обязателен для всех студентов
	\end{enumerate}
		\item Дата выдачи задания: \uline{\thesisStartDate.}
\end{enumerate}

\intervalS%можно удалить пробел

Руководитель ВКР \underline{\hspace*{0.1\textheight} \Supervisor}


%\intervalS%можно удалить пробел

%Консультант\footnote{В случае, если есть консультант, отличный от консультанта по нормоконтролю.}  \uline{\hspace*{0.1\textheight}\ConsultantExtra}


\intervalS%можно удалить пробел

%Консультант по нормоконтролю \uline{\hspace*{0.1\textheight} \ConsultantNorm}%ПОКА НЕ ТРЕБУЕТСЯ, Т.К. ОН У ВСЕХ ПО УМОЛЧАНИЮ

Задание принял к исполнению \uline{\thesisStartDate}

\intervalS%можно удалить пробел

Студент \underline{\hspace*{0.1\textheight}  \Author}



\input{my_folder/task_settings_restore}	% настройки - конец