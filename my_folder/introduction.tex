\chapter*{Введение} % * не проставляет номер
\addcontentsline{toc}{chapter}{Введение} % вносим в содержание

Наша жизнь прочно связана с цифровыми технологиями. Сегодня трудно представить существование без компьютера или интернета. Большинство повседневных действий переводится в онлайн. С каждым годом все больше появляется новых информационных систем, а любая информационная система, будь то приложение или вебсайт, тем или иным способом взаимодействует с данными, которые нужно правильно хранить и обрабатывать. Для взаимодействия с данными большого объема были разработаны базы данных. 

Одним из важнейших этапов разработки любого программного обеспечения является обработка исключительных ситуаций. Такие ситуации не возникают в идеальных условиях, но повсеместно встречаются в нашей жизни. Мы не можем предусмотреть абсолютно все возможные варианты развития событий, но постараться предугадать самые часто встречающиеся ошибки, нам никто не запрещает. 

При разработке программного продукта любой программист задается вопросами разного рода. Что будет с машиной, если неожиданно перестанет поступать питание? А если пользователь выйдет из приложения, не сохранив отчет? Что нужно передать клиенту, если запрашиваемых данных не существует? Продолжать список можно бесконечно, но если мы предполагаем, что такая ситуация может возникнуть, то не лишним будет попробовать защититься от последствий. При некоторых исключительных ситуациях такие последствия могут быть фатальными, стоимость необработанной ошибки в крупных кампаниях может достигать миллионов долларов, а для некоторых сфер, может исчисляться и в человеческих жизнях. Поэтому для разработчиков нужно предоставить максимально удобный и простой способ для обработки исключительных ситуаций. 

Одной из самых популярных систем управления базами данных уже достаточно долгое время является Oracle Database. Она славиться своей надежностью, производительностью, масштабируемостью и безопасностью.

Oracle предоставляет разработчикам баз данных очень гибкий и мощный механизм для работы с ошибками, но он не лишен недостатков, которые могут создать существенные проблемы для групп разработчиков, которые хотят построить систему управления ошибками, обладающую свойствами надежности, содержательности и последовательности.

Данная работа актуально так же потому, что большинство крупных кампаний разрабатывают собственные системы взаимодействия с исключительными ситуациями. Такие разработки являются закрытыми, что усложняет процесс создания новых приложений для баз данных, так как необходимую функциональность требуется разрабатывать с нуля.
%Целью данной работы является изучение существующих способов взаимодействия с исключительными ситуациями %на языке Oracle PL/SQL и выявление недостатков в данном механизме.

Целью данной работы является выявление недостатков в системах обработки ошибок в базах данных Oracle и разработка пакета на языке PL/SQL, нацеленного на исправление данных проблем.

Для достижения поставленной цели необходимо:
\begin{enumerate}
	\item Изучить предоставляемые Oracle механизмы работы с ошибками.
	\item Выявить основные недостатки и неудобства при работе с данной системой.
	\item Проанализировать, полученные на прошлом этапе результаты, и предложить способы исправления данных проблем.
	\item Реализовать пакет на языке PL/SQL, предназначенный для улучшения взаимодействия с существующей системой обработки исключительных ситуаций
	\item Протестировать работу пакета
	\item Сравнить разработанное решение со стандартным способом работы с ошибками
\end{enumerate} 

Предполагается, что если разработать пакет для упрощения взаимодействия с исключительными ситуациями, то написание приложений для баз данных станет более эффективным. 

Объектом исследования является система управления базами данных Oracle, а предметом - исключительные ситуации в языке PL/SQL. 


%% Вспомогательные команды - Additional commands
\newpage % принудительное начало с новой страницы, использовать только в конце раздела
%\clearpage % осуществляется пакетом <<placeins>> в пределах секций
%\newpage\leavevmode\thispagestyle{empty}\newpage % 100 % начало новой строки