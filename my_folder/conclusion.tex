\chapter*{Заключение} \label{ch-conclusion}
\addcontentsline{toc}{chapter}{Заключение}	% в оглавление 

В данной работе были изучены механизмы взаимодействия с исключительными ситуациями, предоставляемые разработчикам приложений в базе данных Oracle на языке PL/SQL. В ходе изучения данной систем были выявлены некоторые недостатки и неудобства, усложняющие процесс создания новых приложений. Основной проблемой являлось необходимость дополнительных действий со стороны команды программистов для контроля над исключительными ситуациями. Возможность упростить данный процесс, послужила поводом для разработки пакета.
 
Были рассмотрены различные способы устранения большинства критических проблем. Из предложенных решений были выбраны самые оптимальные и наиболее эффективные алгоритмы. На основе данных алгоритмов был разработан пакет на языке PL/SQL. Используемые подходы были подробно рассмотрены, была описана обоснованность и целесообразность их использования. Пакет был разработан с обширными возможностями для его дальнейшего расширения, как со стороны автора, так и со стороны пользователей пакета. 

Полученный пакет был тщательно протестирован на выявление ошибок при работе. Тестированию подлежали все элементы пакета, начиная с публичных методов, доступных пользователю, заканчивая скрытыми методами, которые необходимы для выполнения внутренних задач.
 
Данное решение подверглось сравнению со стандартным способом обработки исключительных ситуаций, предоставляемым разработчикам кампанией Oracle. Был выявлен ряд существенных упрощений при разработке с использованием созданного пакета. 


