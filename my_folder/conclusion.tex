\chapter*{Заключение} \label{ch-conclusion}
\addcontentsline{toc}{chapter}{Заключение}	% в оглавление 

%В ходе данной работы были изучены способы и основные подходы при работе с исключительными ситуациями в языке PL/SQL. Было выяснено, что существующий механизм обладает довольно мощным функционалом, но не лишен своих недостатков, которые могут создать трудности при написании приложений для базы данных, а также могут привести к повышению стоимости разработки конечного продукта. Каждая из замеченных проблем была описана, и были предложены способы её исправления.

В ходе данной работы было предложено решение для автоматизации и систематизации обработки ошибок в Oracle PL/SQL. Согласно с выдвинутыми методами, был разработан пакет для взаимодействия с исключительными ситуациями. 
Работа реализованного пакета была тщательно протестирована, было проведено сравнение полученного решения со стандартными методами обработки ошибок. 

